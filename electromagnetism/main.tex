%%%%%%%%%%%%%%%%%%%%          DOCUMENT HEADER          %%%%%%%%%%%%%%%%%%%%
\documentclass[twoside,10pt]{article}
\usepackage{subfigure}
\usepackage{wrapfig}

% page settings
\textwidth=16cm
\oddsidemargin=5mm
\evensidemargin=-5mm

%%%%%%%%%%%%%%%%%%%%      TITLE, AUTHOR, AFFILATION     %%%%%%%%%%%%%%%%%%%%
\title{\Large \bf Electromagnetism}
\date{}

\begin{document}
\maketitle

% \begin{center}
\vspace*{-2cm}
% {\it on behalf of the H1 and ZEUS Collaborations\\}
% \vspace*{0.5cm}
% {\it$^1$DESY, Notkestra{\ss}e 85, 22607 Hamburg, Germany
% }
% % e-mail: }
% \end{center}

%%%%%%%%%%%%%%%%%%%%             ABSTRACT               %%%%%%%%%%%%%%%%%%%%

% \vspace{0.3cm}
% \begin{center}
% {\bf Abstract}\\
% \medskip
% \parbox[t]{10cm}{\footnotesize
% Recent measurements of charm and beauty production at HERA are presented.
% Data are compared to NLO QCD predictions in various schemes which treat differently heavy quark mass.
% Overall, a good description of the data is found.
% Determination of the charm quark mass from reduced charm production cross sections is discussed.
% }
% \end{center}

%%%%%%%%%%%%%%%%%%%%             MAIN TEXT            %%%%%%%%%%%%%%%%%%%%

\section{Electrostatics}

\begin{equation}
{\bf E} = \frac{q}{4 \pi \epsilon_0} \frac{\bf r}{r^3} - \textrm{electric field of a static point charge}
\end{equation}

\begin{equation}
{\bf E} = \sum_i {\bf E}_i - \textrm{superposition principle}
\end{equation}

\begin{equation}
{\rm div} \bf {E} = \rho/\epsilon_0 - \textrm{Gauss's Theorem}
\end{equation}



%%%%%%%%%%%%%%%%%%%%             BIBLIOGRAPHY               %%%%%%%%%%%%%%%%%%%%


\begin{thebibliography}{99}

\bibitem{beauty_dielectron} Irodov

\end{thebibliography}

\end{document}
